\documentclass{report}

\title{\huge Abdullah's Document}
\date{2020-02-04}
\author{Abdullah Karbanee}

\begin{document}
	\maketitle
	\newpage
	\chapter{The Beginning}

Norton woke with a start to face the first light. Rain tapped against the glass.
It was January the fifth.
He looked across a table on which a night-light had guttered into a pool of
water, at the other bed. Francis Morton was still asleep, and Peter lay down
again with his eyes on his brother. It amused him to imagine it was himself whom
he watched, the same hair, the same eyes, the same lips and line of cheek. But
the thought palled, and the mind went back to the fact which lent the day
importance. It was the fifth of January. He could hardly believe a year had
passed since Mrs Henne-Falcon had given her last children's party.
Francis turned suddenly upon his back and threw an arm across his face, blocking
his mouth. Peter's heart began to beat fast, not with pleasure now but with
uneasiness. He sat up and called across the table, "Wake up." Francis's
shoulders shook and he waved a clenched fist in the air, but his eyes remained
closed. To Peter Morton the whole room seemed to darken, and he had the
impression of a great bird swooping. He cried again, "Wake up," and once more
there was silver light and the touch of rain on the windows.
Francis rubbed his eyes. "Did you call out?"' he asked.
"You are having a bad dream," Peter said. Already experience had taught him how
far their minds reflected each other. But he was the elder, by a matter of
minutes, and that brief extra interval of light, while his brother still
struggled in pain and darkness, had given him self-reliance and an instinct of
protection towards the other who was afraid of so many things.
"I dreamed that I was dead," Francis said.
"What was it like?"' Peter asked.
"I can't remember," Francis said.
"You dreamed of a big bird."
"Did I?"
The two lay silent in bed facing each other, the same green eyes, the same nose
tilting at the tip, the same firm lips, and the same premature modelling of the
chin. The fifth of January, Peter thought again, his mind drifting idly from the
image of cakes to the prizes which might be won. Egg-and-spoon races, spearing
apples in basins of water, blind man's buff.
"I don't want to go," Francis said suddenly. "I suppose Joyce will be there …
Mabel Warren." Hateful to him, the thought of a party shared with those two.
They were older than he. Joyce was eleven and Mabel Warren thirteen. The long
pigtails swung superciliously to a masculine stride. Their sex humiliated him,
as they watched him fumble with his egg, from under lowered scornful lids. And
last year … he turned his face away from Peter, his cheeks scarlet.
"What's the matter?"' Peter asked.
"Oh, nothing. I don't think I'm well. I've got a cold. I oughtn't to go to the
party."
Peter was puzzled. "But Francis, is it a bad cold?"
"It will be a bad cold if I go to the party. Perhaps I shall die."
"Then you mustn't go," Peter said, prepared to solve all difficulties with one
plain sentence, and Francis let his nerves relax, ready to leave everything to
Peter. But though he was grateful he did not turn his face towards his brother.
His cheeks still bore the badge of a shameful memory, of the game of hide and
seek last year in the darkened house, and of how he had screamed when Mabel
Warren put her hand suddenly upon his arm. He had not heard her coming. Girls
were like that. Their shoes never squeaked. No boards whined under the tread.
They slunk like cats on padded claws.
When the nurse came in with hot water Francis lay tranquil leaving everything to
Peter. Peter said, "Nurse, Francis has got a cold."
The tall starched woman laid the towels across the cans and said, without
turning, "The washing won't be back till tomorrow. You must lend him some of
your handkerchiefs."
"But, Nurse," Peter asked, "hadn't he better stay in bed?"
"We'll take him for a good walk this morning," the nurse said. "Wind'll blow
away the germs. Get up now, both of you," and she closed the door behind her.
"I'm sorry," Peter said. "Why don't you just stay in bed? I'll tell mother you
felt too ill to get up." But rebellion against destiny was not in Francis's
power. If he stayed in bed they would come up and tap his chest and put a
thermometer in his mouth and look at his tongue, and they would discover he was
malingering. It was true he felt ill, a sick empty sensation in his stomach and
a rapidly beating heart, but he knew the cause was only fear, fear of the party,
fear of being made to hide by himself in the dark, uncompanioned by Peter and
with no night-light to make a blessed breach.
"No, I'll get up," he said, and then with sudden desperation, "But I won't go to
Mrs Henne-Falcon's party. I swear on the Bible I won't." Now surely all would be
well, he thought. God would not allow him to break so solemn an oath. He would
show him a way. There was all the morning before him and all the afternoon until
four o'clock. No need to worry when the grass was still crisp with the early
frost. Anything might happen. He might cut himself or break his leg or really
catch a bad cold. God would manage somehow.

	\chapter{Confidence}

He had such confidence in God that when at breakfast his mother said, "I hear
you have a cold, Francis," he made light of it. "We should have heard more about
it," his mother said with irony, "if there was not a party this evening," and
Francis smiled, amazed and daunted by her ignorance of him.
His happiness would have lasted longer if, out for a walk that morning, he had
not met Joyce. He was alone with his nurse, for Peter had leave to finish a
rabbit-hutch in the woodshed. If Peter had been there he would have cared less;
the nurse was Peter's nurse also, but now it was as though she were employed
only for his sake, because he could not be trusted to go for a walk alone. Joyce
was only two years older and she was by herself.
She came striding towards them, pigtails flapping. She glanced scornfully at
Francis and spoke with ostentation to the nurse. "Hello, Nurse. Are you bringing
Francis to the party this evening? Mabel and I are coming." And she was off
again down the street in the direction of Mabel Warren's home, consciously alone
and self-sufficient in the long empty road.
"Such a nice girl," the nurse said. But Francis was silent, feeling again the
jump-jump of his heart, realizing how soon the hour of the party would arrive.
God had done nothing for him, and the minutes flew.
They flew too quickly to plan any evasion, or even to prepare his heart for the
coming ordeal. Panic nearly overcame him when, all unready, he found himself
standing on the doorstep, with coat-collar turned up against a cold wind, and
the nurse's electric torch making a short trail through the darkness. Behind him
were the lights of the hall and the sound of a servant laying the table for
dinner, which his mother and father would eat alone. He was nearly overcome by
the desire to run back into the house and call out to his mother that he would
not go to the party, that he dared not go. They could not make him go. He could
almost hear himself saying those final words, breaking down for ever the barrier
of ignorance which saved his mind from his parents' knowledge. "I'm afraid of
going. I won't go. I daren't go. They'll make me hide in the dark, and I'm
afraid of the dark. I'll scream and scream and scream."
 
\end{document}
